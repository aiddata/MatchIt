\subsubsection{Syntax}
\begin{verbatim}
> m.out <- matchit(formula, data, method = "nearest", verbose = FALSE, ...)
\end{verbatim}

\subsubsection{Arguments}

\paragraph{Arguments for All Matching Methods}

\begin{itemize}

\item \texttt{formula}: formula used to calculate the distance measure
  for matching (e.g., the propensity score model).  It takes the usual syntax of R formulas, {\tt treat
    \~\ x1 + x2}, where {\tt treat} is a binary treatment indicator,
  and {\tt x1} and {\tt x2} are the pre-treatment covariates. Both the
  treatment indicator and pre-treatment covariates must be contained
  in the same data frame, which is specified as {\tt data} (see
  below).  All of the usual R syntax for formulas work here. For
  example, {\tt x1:x2} represents the first order interaction term
  between {\tt x1} and {\tt x2}, and {\tt I(x1 \^\ 2)} represents the
  square term of {\tt x1}. See {\tt help(formula)} for details.
  
\item \texttt{data}: the data frame containing the variables called in
  {\tt formula}.  
  
\item \texttt{method}: the matching method (default =
  \texttt{"nearest"}, nearest neighbor matching).  Currently,
  \texttt{"exact"} (exact matching), \texttt{"full"} (full matching),
  \texttt{"nearest"} (nearest neighbor matching), \texttt{"optimal"}
  (optimal matching), \texttt{"subclass"} (subclassification), and
  \texttt{"genetic"} (genetic matching) are available. Note that
  within each of these matching methods, \MatchIt\ offers a variety of
  options. See below for more details.
  
\item \texttt{verbose}: a logical value indicating whether to print
  the status of the matching algorithm (default = \texttt{FALSE}).
\end{itemize}


\paragraph{Additional Arguments for Specification of
  Distance Measures}
\label{subsubsec:inputs-all}

The following arguments specify distance measures that are used for
matching methods. These arguments apply to all matching methods {\it
  except exact matching}.

\begin{itemize}
  
\item \texttt{distance}: the method used to estimate the distance
  measure (default = {\tt "logit"}, logistic regression) or a
  numerical vector of user's own distance measure.  Before using any
  of these techniques, it is best to understand the theoretical
  groundings of these techniques and to evaluate the results.  Most of
  these methods (such as logistic or probit regression) define the
  distance by first estimating the propensity score, defined as the
  probability of receiving treatment, conditional on the covariates.
  Available methods include:
  \begin{itemize}
  \item {\tt "mahalanobis"}: the Mahalanobis distance measure.
  \item binomial generalized linear models with one of the following
    link functions:
    \begin{itemize}
    \item \texttt{"logit"}: logistic link 
    \item {\tt "linear.logit"}: logistic link with linear propensity
      score\footnote{The linear propensity scores are obtained by
        transforming back onto a linear scale.}
    \item \texttt{"probit"}: probit link
    \item {\tt "linear.probit"}: probit link with linear propensity
      score
    \item {\tt "cloglog"}: complementary log-log link
    \item {\tt "linear.cloglog"}: complementary log-log link with linear
      propensity score
    \item {\tt "log"}: log link
    \item {\tt "linear.log"}: log link with linear propensity score
    \item {\tt "cauchit"} Cauchy CDF link
    \item {\tt "linear.cauchit"} Cauchy CDF link with linear propensity
      score
    \end{itemize}
  \item Choose one of the following generalized additive models (see
    {\tt help(gam)} for more options).
    \begin{itemize}
    \item \texttt{"GAMlogit"}: logistic link
    \item {\tt "GAMlinear.logit"}: logistic link with linear propensity
      score
    \item \texttt{"GAMprobit"}: probit link
    \item {\tt "GAMlinear.probit"}: probit link with linear propensity
      score
    \item {\tt "GAMcloglog"}: complementary log-log link 
    \item {\tt "GAMlinear.cloglog"}: complementary log-log link with
      linear propensity score
    \item {\tt "GAMlog"}: log link
    \item {\tt "GAMlinear.log"}: log link with linear propensity score,
    \item {\tt "GAMcauchit"}: Cauchy CDF link
    \item {\tt "GAMlinear.cauchit"}: Cauchy CDF link with linear
      propensity score
\end{itemize} 
\item \texttt{"nnet"}: neural network model.  See {\tt help(nnet)} for
  more options.
\item \texttt{"rpart"}: classification trees.  See {\tt help(rpart)}
  for more options.
  \end{itemize}
  
\item \texttt{distance.options}: optional arguments for estimating the
  distance measure. The input to this argument should be a list.  For
  example, if the distance measure is estimated with a logistic
  regression, users can increase the maximum IWLS iterations by
  \texttt{distance.options = list(maxit = 5000)}.  Find additional
  options for general linear models using {\tt help(glm)} or {\tt
    help(family)}, for general additive models using {\tt help(gam)},
  for neutral network models {\tt help(nnet)}, and for classification
  trees {\tt help(rpart)}.
  
\item \texttt{discard}: specifies whether to discard units that fall
  outside some measure of support of the distance measure (default =
  \texttt{"none"}, discard no units).  Discarding units may change the
  quantity of interest being estimated by changing the observations left in the analysis. 
  Enter a logical vector
  indicating which unit should be discarded or choose from the
  following options:
  \begin{itemize}
  \item \texttt{"none"}: no units will be discarded before matching.
    Use this option when the units to be matched are substantially
    similar, such as in the case of matching treatment and control
    units from a field experiment that was close to (but not fully)
    randomized (e.g., \citealt{Imai05}), when caliper matching will
    restrict the donor pool, or when you do not wish to change the
    quantity of interest and the parametric methods to be used
    post-matching can be trusted to extrapolate.
  \item \texttt{"hull.both"}: all units that are not within the convex
    hull will be discarded.  See \citet{KinZen06,KinZen07} for
    information about the convex hull in this context and as a measure
    of model dependence.
  \item \texttt{"both"}: all units (treated and control) that are
    outside the support of the distance measure will be discarded.
  \item \texttt{"hull.control"}: only control units that are not
    within the convex hull of the treated units will be discarded.  
  \item \texttt{"control"}: only control units outside the support of
    the distance measure of the treated units will be discarded.  Use
    this option when the average treatment effect on the treated is of
    most interest and when you are unwilling to discard
    non-overlapping treatment units (which would change the quantity
    of interest).
  \item \texttt{"hull.treat"}: only treated units that are not within
    the convex hull of the control units will be discarded. 
  \item \texttt{"treat"}: only treated units outside the support of
    the distance measure of the control units will be discarded.  Use
    this option when the average treatment effect on the control units
    is of most interest and when unwilling to discard control units.
  \end{itemize}
  
\item \texttt{reestimate}: If {\tt FALSE} (default), the model for the
  distance measure will not be re-estimated after units are discarded.
  The input must be a logical value.  Re-estimation may be desirable
  for efficiency reasons, especially if many units were discarded and
  so the post-discard samples are quite different from the original
  samples.
  
\end{itemize}

\paragraph{Additional Arguments for Subclassification}
\label{subsubsec:inputs-subclass}

\begin{itemize}
\item \texttt{sub.by}: criteria for subclassification.  Choose from:
  \texttt{"treat"} (default), the number of treatment units;
  \texttt{"control"}, the number of control units; or \texttt{"all"},
  the total number of units.  Changing the default will likely also
  signal a change in your quantity of interest from the average
  treatment effect on the treated to other quantities.
\item \texttt{subclass}: either a scalar specifying the number of
  subclasses, or a vector of probabilities bounded between 0 and 1,
  which create quantiles of the distance measure using the units in
  the group specified by \texttt{sub.by} (default = \texttt{subclass =
    6}).  
\end{itemize}

\paragraph{Additional Arguments for Nearest Neighbor Matching}
\label{subsubsec:inputs-nearest}

\begin{itemize}
\item \texttt{m.order}: the order in which to match treatment units
  with control units.
  \begin{itemize}
  \item {\tt "largest"} (default): matches from the largest value of
    the distance measure to the smallest.
  \item {\tt "smallest"}: matches from the smallest value of the
    distance measure to the largest.
  \item {\tt "random"}: matches in random order.
  \end{itemize}
\item \texttt{replace}: logical value indicating whether each control
  unit can be matched to more than one treated unit (default = {\tt
    replace = FALSE}, each control unit is used at most once -- i.e.,
  sampling without replacement). For matching with replacement, use
  \texttt{replace = TRUE}.  After matching with replacement, the weights
  can be used to reflect the frequency with which each control unit was matched.
\item \texttt{ratio}: the number of control units to match to each
  treated unit (default = {\tt 1}).  If matching is done without
  replacement and there are fewer control units than {\tt ratio} times
  the number of eligible treated units (i.e., there are not enough
  control units for the specified method), then the higher ratios will
  have \texttt{NA} in place of the matching unit number in
  \texttt{match.matrix}.
\item \texttt{exact}: variables on which to perform exact matching
  within the nearest neighbor matching (default = {\tt NULL}, no exact
  matching).  If \texttt{exact} is specified, only matches that
  exactly match on the covariates in \texttt{exact} will be allowed.
  Within the matches that match on the variables in \texttt{exact},
  the match with the closest distance measure will be chosen.
  \texttt{exact} should be entered as a vector of variable names
  (e.g., \texttt{exact = c("X1", "X2")}).
\item \texttt{caliper}: the number of standard deviations of the
  distance measure within which to draw control units (default = {\tt
    0}, no caliper matching).  If a caliper is specified, a control
  unit within the caliper for a treated unit is randomly selected as
  the match for that treated unit.  If \texttt{caliper != 0}, there
  are two additional options:
  \begin{itemize} 
  \item \texttt{calclosest}: whether to take the nearest available
    match if no matches are available within the \texttt{caliper}
    (default = {\tt FALSE}).
  \item \texttt{mahvars}: variables on which to perform
    Mahalanobis-metric matching within each caliper (default = {\tt
      NULL}).  Variables should be entered as a vector of variable
    names (e.g., \texttt{mahvars = c("X1", "X2")}).  If
    \texttt{mahvars} is specified without \texttt{caliper}, the
    caliper is set to 0.25.
  \end{itemize}
\item \texttt{subclass} and \texttt{sub.by}: See the options for
  subclassification for more details on these options.  If a
  \texttt{subclass} is specified within \texttt{method = "nearest"},
  the matched units will be placed into subclasses after the nearest
  neighbor matching is completed.
\end{itemize}

\paragraph{Additional Arguments for Optimal Matching}
\label{subsubsec:inputs-optimal}

\begin{itemize}
\item {\tt ratio}: the number of control units to be matched to each
  treatment unit (default = {\tt 1}).
\item {\tt ...}: additional inputs that can be passed to the {\tt
    fullmatch()} function in the {\tt optmatch} package. See {\tt
    help(fullmatch)} or
  \hlink{http://www.stat.lsa.umich.edu/\~{}bbh/optmatch.html}{http://www.stat.lsa.umich.edu/~bbh/optmatch.html}
  for details.
\end{itemize}

\paragraph{Additional Arguments for Full Matching}
\label{subsubsec:inputs-full}

See {\tt help(fullmatch)} (part of this information is copied below)
or
\hlink{http://www.stat.lsa.umich.edu/\~{}bbh/optmatch.html}{http://www.stat.lsa.umich.edu/~bbh/optmatch.html}
for details.

\begin{itemize}
\item {\tt min.controls}: The minimum ratio of controls to treatments
  that is to be permitted within a matched set: should be nonnegative
  and finite.  If {\tt min.controls} is not a whole number, the
  reciprocal of a whole number, or zero, then it is rounded down to
  the nearest whole number or reciprocal of a whole number.
  
\item {\tt max.controls}: The maximum ratio of controls to treatments
  that is to be permitted within a matched set: should be positive and
  numeric. If {\tt max.controls} is not a whole number, the reciprocal
  of a whole number, or {\tt Inf}, then it is rounded up to the
  nearest whole number or reciprocal of a whole number.
  
\item {\tt omit.fraction}: Optionally, specify what fraction of
  controls or treated subjects are to be rejected.  If {\tt
    omit.fraction} is a positive fraction less than one, then {\tt
    fullmatch()} leaves up to that fraction of the control reservoir
  unmatched.  If {\tt omit.fraction} is a negative number greater than
  $-1$, then {\tt fullmatch()} leaves up to $|{\rm omit.fraction}|$ of
  the treated group unmatched.  Positive values are only accepted if
  ${\rm max.controls} >= 1$; negative values, only if ${\rm
    min.controls} <= 1$.  If {\tt omit.fraction} is not specified,
  then only those treated and control subjects without permissible
  matches among the control and treated subjects, respectively, are
  omitted.

\item {\tt ...}: Additional inputs that can be passed to the {\tt
    fullmatch()} function in the {\tt optmatch} package.
\end{itemize}

\paragraph{Additional Arguments for Genetic Matching}
\label{subsubsec:inputs-genetic}

The available options are listed below.
\begin{itemize}
\item {\tt ratio}: the number of control units to be matched to each
  treatment unit (default = {\tt 1}).
\item {\tt ...}: additional minor inputs that can be passed to the
  {\tt GenMatch()} function in the {\tt Matching} package. See {\tt
    help(GenMatch)} or\\
  \hlink{http://sekhon.polisci.berkeley.edu/library/Matching/html/GenMatch.html}{http://sekhon.polisci.berkeley.edu/library/Matching/html/GenMatch.html}
  for details.
\end{itemize}

\subsubsection{Output Values}
\label{sec:outputs}

Regardless of the type of matching performed, the \texttt{matchit}
output object contains the following elements:\footnote{When
  inapplicable or unnecessary, these elements may equal {\tt NULL}.
  For example, when exact matching, {\tt match.matrix = NULL}.}

\begin{itemize}
\item \texttt{call}: the original {\tt matchit()} call.
  
\item \texttt{formula}: the formula used to specify the model for
  estimating the distance measure.
  
\item \texttt{model}: the output of the model used to estimate
  the distance measure.  \texttt{summary(m.out\$model)} will give the
  summary of the model where \texttt{m.out} is the output object from
  \texttt{matchit()}.
  
\item \texttt{match.matrix}: an $n_1 \times$ \texttt{ratio} matrix
  where:
  \begin{itemize}
  \item the row names represent the names of the treatment units
    (which match the row names of the data frame specified in
    \texttt{data}).
  \item each column stores the name(s) of the control unit(s) matched
    to the treatment unit of that row. For example, when the
    \texttt{ratio} input for nearest neighbor or optimal matching is
    specified as 3, the three columns of \texttt{match.matrix}
    represent the three control units matched to one treatment unit).
  \item \texttt{NA} indicates that the treatment unit was not matched.
  \end{itemize}
  
\item \texttt{discarded}: a vector of length $n$ that displays whether
  the units were ineligible for matching due to common support
  restrictions.  It equals \texttt{TRUE} if unit $i$ was discarded,
  and it is set to \texttt{FALSE} otherwise.
  
\item \texttt{distance}: a vector of length $n$ with the estimated
  distance measure for each unit.
  
\item \texttt{weights}: a vector of length $n$ with the weights
  assigned to each unit in the matching process.  Unmatched units have
  weights equal to $0$. Matched treated units have weight $1$.  Each
  matched control unit has weight proportional to the number of
  treatment units to which it was matched, and the sum of the control
  weights is equal to the number of uniquely matched control units.
 
\item \texttt{subclass}: the subclass index in an ordinal scale from 1
  to the total number of subclasses as specified in \texttt{subclass}
  (or the total number of subclasses from full or exact matching).
  Unmatched units have \texttt{NA}.
  
\item \texttt{q.cut}: the subclass cut-points that classify the
  distance measure.
  
\item \texttt{treat}: the treatment indicator from \texttt{data} (the
  left-hand side of \texttt{formula}).
 
\item \texttt{X}: the covariates used for estimating the distance
  measure (the right-hand side of \texttt{formula}).  When applicable,
  \texttt{X} is augmented by covariates contained in \texttt{mahvars}
  and \texttt{exact}.
\end{itemize}


%%% Local Variables: 
%%% mode: latex
%%% TeX-master: "matchit"
%%% End: 
