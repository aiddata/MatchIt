\section{Preprocessing via Matching}
\label{sec:matching}

\subsection{Quick Overview}

The main command \texttt{matchit()} implements the matching
procedures.  A general syntax is:
\begin{verbatim}
> m.out <- matchit(treat ~ x1 + x2, data = mydata)
\end{verbatim}
where {\tt treat} is the dichotomous treatment variable, and {\tt x1}
and {\tt x2} are pre-treatment covariates, all of which are contained
in the data frame {\tt mydata}.  The dependent variable (or variables)
may be included in \texttt{mydata} for convenience but is never used
by \MatchIt\ or included in the formula.  This command creates the
\MatchIt\ object called \texttt{m.out}.  Name the output object to see
a quick summary of the results:
\begin{verbatim}
> m.out
\end{verbatim}

\subsection{Examples}

To run any of the examples below, you first must load the library and
and data:
\begin{verbatim}
> library(MatchIt)
> data(lalonde)
\end{verbatim}

Our example data set is a subset of the job training program analyzed
in \citet{lalonde86} and \citet{DehWah99}. \MatchIt\ includes a
subsample of the original data consisting of the National Supported
Work Demonstration (NSW) treated group and the comparison sample from
the Population Survey of Income Dynamics (PSID).\footnote{This data
  set, \texttt{lalonde}, was created using NSWRE74$\_$TREATED.TXT and
  CPS3$\_$CONTROLS.TXT from
  http://www.columbia.edu/$\sim$rd247/nswdata.}  The variables in this
data set include participation in the job training program
(\texttt{treat}, which is equal to 1 if participated in the program,
and 0 otherwise), age ({\tt age}), years of education ({\tt educ}),
race (\texttt{black} which is equal to 1 if black, and 0 otherwise;
\texttt{hispan} which is equal to 1 if hispanic, and 0 otherwise),
marital status (\texttt{married}, which is equal to 1 if married, 0
otherwise), high school degree (\texttt{nodegree}, which is equal to 1
if no degree, 0 otherwise), 1974 real earnings (\texttt{re74}), 1975
real earnings (\texttt{re75}), and the main outcome variable, 1978
real earnings (\texttt{re78}).

\subsubsection{Exact Matching}
\label{subsubsec:exact}

The simplest version of matching is exact.  This technique matches
\emph{each} treated unit to \emph{all} possible control units with
exactly the same values on all the covariates, forming subclasses such
that within each subclass all units (treatment and control) have the
same covariate values.  Exact matching is implemented in \MatchIt\ 
using \texttt{method = "exact"}.  Exact matching will be done on all
covariates included on the right-hand side of the \texttt{formula}
specified in the \MatchIt\ call.  There are no additional options for
exact matching.  (Exact restrictions on a subset of covariates can
also be specified in nearest neighbor matching; see
Section~\ref{subsubsec:nearest}.)  The following example can be
run by typing {\tt demo(exact)} at the R prompt,
\begin{verbatim}
> m.out <- matchit(treat ~ educ + black + hispan, data = lalonde, 
                   method = "exact")
\end{verbatim}

\subsubsection{Subclassification}
\label{subsubsec:subclass}

When there are many covariates (or some covariates can take a large
number of values), finding sufficient exact matches will often be
impossible.  The goal of subclassification is to form subclasses, such
that in each the distribution (rather than the exact values) of
covariates for the treated and control groups are as similar as
possible.  Various subclassification schemes exist, including the one
based on a scalar distance measure such as the propensity score
estimated using the \texttt{distance} option (see
Section~\ref{subsubsec:inputs-all}).  Subclassification is implemented
in \MatchIt\ using \texttt{method = "subclass"}.

The following example script can be run by typing {\tt demo(subclass)}
at the R prompt,
\begin{verbatim}
> m.out <- matchit(treat ~ re74 + re75 + educ + black + hispan + age, 
                   data = lalonde, method = "subclass")
\end{verbatim}
The above syntax forms 6 subclasses, which is the default number
of subclasses, based on a distance measure (the propensity score) estimated using logistic
regression.  By default, each subclass will have approximately the
same number of treated units.

Subclassification may also be used in conjunction with nearest
neighbor matching described below, by leaving the default of
\texttt{method = "nearest"} but adding the option \texttt{subclass}.
When you choose this option, \MatchIt\ selects matches using nearest neighbor
matching, but
after the nearest neighbor matches are chosen it places them into
subclasses, and adds a variable to the output object indicating subclass membership.


\subsubsection{Nearest Neighbor Matching}
\label{subsubsec:nearest}

Nearest neighbor matching selects the $r$ (default=1) best control
matches for each individual in the treatment group (excluding those
discarded using the \texttt{discard} option).  Matching is done
using a distance measure specified by the {\tt distance} option
(default=logit).  Matches are chosen for each treated unit one at a
time, with the order specified by the \texttt{m.order} command (default=largest to smallest).  
At each matching step we choose the control unit that is not
yet matched but is closest to the treated unit on the distance
measure.

Nearest neighbor matching is implemented in \MatchIt\ using the
\texttt{method = "nearest"} option.  The following example script can
be run by typing {\tt demo(nearest)}:
\begin{verbatim}
> m.out <- matchit(treat ~ re74 + re75 + educ + black + hispan + age, 
                   data = lalonde, method = "nearest")
\end{verbatim}

\subsubsection{Optimal Matching}
\label{subsubsec:optimal}

The default nearest neighbor matching method in \MatchIt\ is
``greedy'' matching, where the closest control match for each treated
unit is chosen one at a time, without trying to minimize a global
distance measure.  In contrast, ``optimal'' matching finds the matched
samples with the smallest average absolute distance across all the
matched pairs.  \citet{GuRos93} find that greedy and optimal matching
approaches generally choose the same sets of controls for the overall
matched samples, but optimal matching does a better job of minimizing
the distance within each pair.  In addition, optimal matching can be
helpful when there are not many appropriate control matches for the
treated units.

Optimal matching is performed with \MatchIt\ by setting \texttt{method
  = "optimal"}, which automatically loads an add-on package called
\texttt{optmatch} \citep{Hansen04}.  The following example can also be
run by typing {\tt demo(optimal)} at the R prompt.  We conduct 2:1 optimal
ratio matching based on the propensity score from the logistic
regression.
\begin{verbatim}
> m.out <- matchit(treat ~ re74 + re75 + age + educ, data = lalonde, 
                   method = "optimal", ratio = 2)
\end{verbatim}

\subsubsection{Full Matching}
\label{subsubsec:full}

Full matching is a particular type of subclassification that forms the
subclasses in an optimal way \citep{Rosenbaum02, Hansen04}.  A fully
matched sample is composed of matched sets, where each matched set
contains one treated unit and one or more controls (or one control
unit and one or more treated units).  As with subclassification, the
only units not placed into a subclass will be those discarded (if a
\texttt{discard} option is specified) because they are outside the
range of common support.  Full matching is optimal in terms of
minimizing a weighted average of the estimated distance measure
between each treated subject and each control subject within each
subclass.

Full matching can be performed with \MatchIt\ by setting
\texttt{method = "full"}.  Just as with optimal matching, we use the
\texttt{optmatch} package \citep{Hansen04}, which automatically loads
when needed.  The following example with full matching (using the
default propensity score based on logistic regression) can also be run
by typing {\tt demo(full)} at the R prompt:
\begin{verbatim}
> m.out <- matchit(treat ~ age + educ + black + hispan + married +
                   nodegree + re74 + re75, data = lalonde, method = "full")
\end{verbatim}

\subsubsection{Genetic Matching}
\label{subsub:genetic}

Genetic matching automates the process of finding a good matching
solution \citep{DiaSek05}.  The idea is to use a genetic search
algorithm to find a set of weights for each covariate such that the a
version of optimal balance is achieved after matching.  As currently
implemented, matching is done with replacement using the matching
method of \citet{AbaImb07} and balance is determined by two univariate
tests, paired t-tests for dichotomous variables and a
Kolmogorov-Smirnov test for multinomial and continuous variables, but
these options can be changed.

Genetic matching can be performed with \MatchIt\ by setting
\texttt{method = "genetic"}, which automatically loads the
\texttt{Matching} \citep{Sekhon04} package.  The following example of
genetic matching (using the estimated propensity score based on
logistic regression as one of the covariates) can also be run by
typing {\tt demo(genetic)}:
\begin{verbatim}
> m.out <- matchit(treat ~ age + educ + black + hispan + married + nodegree + 
                   re74 + re75, data = lalonde, method = "genetic")
\end{verbatim}


\subsubsection{Coarsened Exact Matching}
\label{subsub:cem}

Coarsened Exact Matching (CEM) is a Monotonoic Imbalance Bounding
(MIB) matching method --- which means that the balance between the
treated and control groups is chosen by the user ex ante rather than
discovered through the usual laborious process of checking after the
fact and repeatedly reestimating, and so that adjusting the imbalance
on one variable has no effect on the maximum imbalance of any other.
CEM also strictly bounds through ex ante user choice both the degree
of model dependence and the average treatment effect estimation error,
eliminates the need for a separate procedure to restrict data to
common empirical support, meets the congruence principle, is robust to
measurement error, works well with multiple imputation methods for
missing data, and is extremely fast computationally even with very
large data sets.  CEM also works well for multicategory treatments,
determining blocks in experimental designs, and evaluating extreme
counterfactuals \citep{IacKinPor08}.

CEM can be performed with \MatchIt\ by setting \texttt{method =
  "cem"}, which automatically loads the \texttt{cem} package.  The
following examples of CEM (with automatic coarsening) can also be run
by typing \texttt{demo(cem)}:
\begin{verbatim}
m.out <- matchit(treat ~ age + educ + black + hispan + married + nodegree 
                 + re74 + re75, data = lalonde, method = "cem")
\end{verbatim}

%%% Local Variables: 
%%% mode: latex
%%% TeX-master: "matchit"
%%% End: 
